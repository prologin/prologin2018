\newpage

\section{Contexte}
\subsection{Episode 1 : unnomàlacon}

Cela fait maintenant trois ans que vous êtes employé dans le bureau des
IeN\footnote{\emph{Individus en Noir}\texttrademark.}.
Affecté depuis la moitié de ce temps dans les
bureaux grisâtres du Nord Pas-de-Calais\footnote{La délocalisation affecte tous
les secteurs.}, vous pensiez que votre carrière se limiterait à servir le café à des
quadragénaires pédants. 
Pourtant, ce matin, vous notez une agitation certaine dans les locaux du 4, Rue
de la FIXME, CodePostal, FRANCE. %FIXME : Trouver une adresse véritable dans un coin bien nul
Effectivement, alors que vous naviguiez entre les bureaux Ikea pour déposer un
latte - tiède, avec deux sucres - sur la grosse patte gélatineuse de
Beauté, l'alien aux télécommunications, vous surprenez des cris parvenant directement du bureau du
Directeur, D. Les têtes se dressent, curieuses, d'un peu partout de
l'open space. Malgré votre professionnalisme à toute épreuve, vous ne pouvez
vous empêcher de jetter un oeil, vous aussi. A travers les stores entrouverts
du bureau, vous voyez le directeur et J\footnote{De son vrai nom Jean
Félicitations}, un de vos collègues affectés au terrain\footnote{Promotion dont vous
rêvez chaque soir : ils ont beaucoup plus de RTT.}, engagés dans une discussion animée dont vous entendez
quelques bribes : apparemment, il se passe quelque chose de très grave
impliquant le Pôle Sud, des aliens, et le menu de la cantine.

Le ton monte, et inquiet, vous fixez maintenant franchement les deux hommes -
cela fait maintenant trop longtemps que le cuisinier essaye de tous vous
empoisonner - quand ceux-ci se détournent brusquement vers vous. Ils vous
regardent, vous les regardez, puis vous rappelant que vous avez un travail à
faire, vous récupérez vaillamment votre plateau et vous dirigez vers le bureau
avec un courage que vos collègues saluent en murmurant. Lorsque vous entrez,
les deux hommes vous observent avec une perplexité évidente, mais vous ne vous démontez pas.

\begin{itemize}
\item[-] Un petit café ?
\item[-] Tiède, avec deux sucres, lâche D d'un ton égal.
\item[-] Ça vous dirait, vous, de diriger une équipe ? Vous demande J.
\item[-] Ah ben oui, pourquoi pas.
\end{itemize}

Silence.

\begin{itemize}
\item[-] Tout de même, ça ne suffira pas, observe le directeur.
\end{itemize}

J ne répond pas tout de suite, l'air pensif. Mal à l'aise, vous dansez d'un
pied à l'autre, attendant soit qu'on vous renvoie, soit une marque visible de
votre promotion soudaine. Votre collègue parcourt la salle des yeux, l'air de
chercher l'inspiration et brusquement, son regard s'arrête sur la pile de
courrier - essentiellement de la pub et des menus de restaurants chinois -
qui s'entasse dans la poubelle et il sourit.

\begin{itemize}
\item[-] J'ai une idée.
\end{itemize}

\subsection{Episode 2 : unautrenomàlacon}

C'est vrai que lorsqu'il avait pioché cette affiche recouverte de
manchots\footnote{Et non pas de pingouins.} de la poubelle, vous n'étiez pas
très convaincu. Beauté avait même ajouté sous sa barbe et ses multiples
mentons que c'était un peu limite d'employer des étudiants, non, vous ne
trouvez pas ?
Ces considérations éthiques avaient été soigneusement ignorées par la
direction, et voilà comment vous vous êtes retrouvés quelque part dans la
banlieue parisienne à assiéger l'amphithéâtre où se trouvait les futures
recrues.

Vous aviez rejoint très tôt le matin, au moins 9h, l'équipe des PiB\footnote{\emph{Prologin in
Black}\texttrademark}, sous-équipe du sous-sous-secteur du Kremlin-Bicêtre,
\st{adorable} petite ville \st{bucolique} d'Île-de-France. Leur président,
B, avait salué votre arrivée avec un cynisme admirable et des remarques sur
votre incompétence certaine, mais vous ne vous êtes pas démonté : vous
comptiez bien montrer à vos collègues, là-haut dans le Nord, que vous en êtes
capable.

Lors d'un rendez-vous précipité dans les locaux exigus des PiB, on vous avait
expliqué la situation dramatique dans laquelle se trouvait le monde.
\guillemotleft \textbf{Tout a commencé dans le Département Administratif des Déplacements
Aliens.}\guillemotright \ déclara B en picorant son kebab, qu'il disait préférer
au plus traditionnel croissant matinal. 

FIXME

Pour patienter, vous attrapez le petit carnet que vous devez distribuer aux
recrues, et parcourrez les instructions.
