\newpage

\section{Contexte}
\subsection{Épisode 1 : Crise}
\hfill \textit{Maubeuge, Nord, Jeudi 17 Mai 2018}

Cela fait maintenant trois ans que vous êtes employé dans le bureau des
IeN\footnote{\emph{Individus en Noir}\texttrademark.}.  Affecté depuis la moitié
de ce temps dans les bureaux grisâtres du Nord Pas-de-Calais\footnote{La
délocalisation affecte tous les secteurs.}, vous pensiez que votre carrière se
limiterait à servir le café à des quadragénaires pédants. 

Pourtant, ce matin, vous notez une agitation certaine dans les locaux du 4, Rue
du Trieu Mouton, Maubeuge, 59600, FRANCE.  Effectivement, alors que vous
naviguiez entre les bureaux Ikea pour déposer un latte - tiède, avec deux sucres
- sur la grosse patte gélatineuse de Beauté, l'alien aux télécommunications,
vous surprenez des cris provenant directement du bureau du Directeur, D. Les
têtes se dressent, curieuses, d'un peu partout de l'open space. Malgré votre
professionnalisme à toute épreuve, vous ne pouvez vous empêcher de jeter un œil,
vous aussi. À travers les stores entrouverts du bureau, vous voyez le directeur
et J\footnote{De son vrai nom Joseph Marchand.}, un de vos collègues affectés au
terrain\footnote{Promotion dont vous rêvez chaque soir : ils ont beaucoup plus
de RTT.}, engagés dans une discussion animée dont vous entendez quelques bribes
: apparemment, il se passe quelque chose de très grave impliquant le Pôle Sud,
des aliens, et le menu de la cantine.

Le ton monte, et inquiet, vous fixez franchement les deux hommes - cela fait
maintenant trop longtemps que le cuisinier essaye de tous vous empoisonner -
quand ceux-ci se détournent brusquement vers vous. Ils vous regardent, vous les
regardez, puis vous vous voyiez soudainement convoqué par un signe sec du
Directeur. Avec un courage que vos collègues saluent en murmurant, vous entrez.

\begin{itemize}
\item[-] Un petit café ?
\item[-] Ça vous dirait de diriger une équipe ?
\item[-] Ah ben oui, pourquoi pas.
\end{itemize}

Silence.

\begin{itemize}
\item[-] Bon, j'ai une idée.
\end{itemize}

Et sur cette déclaration étonnante, J plonge les mains dans la poubelle et en
extirpe une affiche recouverte de manchots\footnote{Et non pas de pingouins.},
où s'étale en grand les mots suivants : \emph{Prologin}.

\subsection{Épisode 2 : Les PiB}
\hfill \textit{Kremlin-Bicêtre, Île-de-France, Samedi 19 Mai 2018}

Voilà comment vous vous êtes retrouvé dans le bureau des
PiB\footnote{\emph{Prologin in Black}\texttrademark}, sous-équipe du
sous-sous-secteur du Kremlin-Bicêtre.

Le Président, B, vous explique la situation en picorant son kebab, qu'il affirme
préférer au traditionnel croissant matinal.

\begin{itemize}
    \item[-] Le Département Administratif des Déplacements Aliens est formel :
        une large invasion alien se prépare.
    \item[-] Mais pourquoi ?
    \item[-] On ne le sait pas vraiment.\footnote{D'après les rumeurs, une
        sombre histoire de clafoutis.}
\end{itemize}
B balaye toutes vos protestations dubitatives d'un mouvement ample de la main.
\begin{itemize}
    \item[-] En tout cas, une chose est certaine : un groupe d'alien débarquera
        en repérage au Pôle Sud. Il faut absolument en capturer avant qu'ils ne
        repartent pour déterminer quel est leur plan et empêcher l'invasion.
        C'est pour cela que nous avons décidé de mettre nos meilleurs agents
        dessus : les Manchots.
    \item[-] Les quoi ?
    \item[-] Les manchots.
    \item[-] Pardon ?
\end{itemize}

B soupira et tire de sa poche une petite photographie froissée. Vous discernez
dessus une silhouette replète, ronde et toute douce, ornée d'une paire de
lunettes de soleil sur son adorable petit bec.

\begin{itemize}
    \item[-] Aaaaw.
    \item[-] Détrompez-vous, ils sont redoutables. Leur vitesse n'a aucun égal
        sur la glace.
    \item[-] Bon, dans ce cas, pourquoi on est là ?
    \item[-] Les manchots ont un problème majeur, explique B en ajustant ses
        lunettes. Ils sont très, très stupides. Ils auront besoin de quelqu'un
        pour leur expliquer quoi faire.
\end{itemize}

Et d'un geste dramatique, B montre de la main les recrues pullulant dans la
cour.

\subsection{Épisode 3 : Les joies de l'administration française}
\hfill \textit{Pôle Sud, Samedi 19 Mai 2018}

Vous sortez vos jumelles et les pointez droit vers l'Iceberg - oui, celui-ci -
jusqu'à retrouver les petits points noirs : les quatre manchots qui composent
votre équipe.
Cependant, vous déchantez vite. Un peu plus haut, vous repérez quatre petits
points noirs supplémentaires : les aliens, déjà !? Non. D'autres
pingouins\footnote{Pardon. D'autres \emph{manchots}.}. Mais d'où ?

Irrité, vous appelez aussitôt votre supérieur hiérarchique direct.

\begin{itemize}
    \item[-] À quoi ça rime ? Pourquoi on est deux équipes sur l'Iceberg ?
    \item[-] Euuuuuh... Attendez.
\end{itemize}

Petite musique d'ascenceur.

\begin{itemize}
    \item[-] Bah en fait, euh, ben, il y a eu une erreur.
    \item[-] Une erreur ?
    \item[-] On a peut-être, ou peut-être pas, envoyé deux équipes au même
        endroit...\footnote{Cf. le titre}
\end{itemize}

Vous raccrochez, et réfléchissez. Que devriez-vous faire ? Laisser l'autre
équipe s'en occuper, quitte à perdre votre travail ? Ou profiter de cette
compétition pour briller aux yeux des PiB ? Le choix est vôtre.\footnote{En
fait, si vous êtes toujours là, vous n'avez plus vraiment le choix.}
