%!TEX TS-program = xelatex
%!TEX encoding = UTF-8 Unicode

\documentclass[a4paper,12pt,notitle,noheader,nofooter]{prologin}
\edition{2018}

\usepackage{fontspec}
\usepackage[french]{babel}
\usepackage{multicol} % Uniquement car 2 partenaires
\usepackage{framed}
\usepackage{tikz}
\usepackage{prologin-colors}
\definecolor{shadecolor}{gray}{.75}
\usepackage{hyperref}

\newenvironment{encadre}{\begin{shaded}}{\end{shaded}}
\newcommand{\stagename}{Prologin 2018}

%\title{Rapport du jury}
%\subtitle{de la finale \stagename}
\geometry{tmargin=3.4cm, bmargin=2cm, lmargin=2.6cm, rmargin=2.6cm}

\begin{document}
\begin{titlepage}
\newgeometry{tmargin=5cm, lmargin=4cm, rmargin=4cm}
\begin{center}
\includegraphics[width=10cm]{../prologin2018}

\vspace{2.4cm}
{
 \Huge \textsf{Concours National d'Informatique}\\
 \vspace{1em}
 \Large Rapport du jury de la finale Prologin 2018
 \vspace{0.5cm}
}
\end{center}
\textit{Membres du jury :}
\begin{center}
\begin{tabular}{l r}
Kaci \textsc{Adjou} & président Prologin 2018 \\
Thibault \textsc{Allançon} & responsable du sujet Prologin 2018 \\
Sacha \textsc{Delanoue} & membre de Prologin \\
Paul \textsc{Guénézan} & membre de Prologin \\
Sébastien \textsc{Hémon} & professeur de mathématiques à EPITA \\
Héloïse \textsc{Vallerio} & ancienne présidente de Prologin
\end{tabular}
\end{center}
\textit{Assistants à la relecture des soumissions :}
\begin{center}
Florian \textsc{Amsallem}, Aliona \textsc{Dangla}, \\
Antoine \textsc{Pietri}, Marc \textsc{Schmitt}, \\
Alexandre \textsc{Talon}

\end{center}
\quad membres de Prologin.
\end{titlepage}

\newgeometry{bmargin=2cm}

Sur les codes des dix meilleurs champions, 7 étaient codés en C++ et 3 en
Python. Les codes allaient de 360 à 1300 lignes, et étaient composés de 620
lignes en moyenne. Cette année, nous avons pu observer peu de diversité du point
de vue des langages : 45 champions en Python, 42 en C++, 3 en C, 1 en Rust et 1
en Java. À noter que le support de Rust est nouveau cette année, un grand merci
à Malo pour avoir implémenté soi-même ce support dans notre
infrastructure\footnote{\url{https://github.com/prologin/stechec2/}} !

\bigbreak

Les candidats ont su exploiter les différentes stratégies et approches
qu'offrait le sujet de cette année, tant d'un point de vue défensif qu'offensif.
Nous regrettons cependant d'avoir vu autant de candidats se focaliser sur
l'aspect temporel des tours afin de réaliser un maximum de calculs dans la
limite de temps autorisée. La plupart se sont rendu compte durant les tournois
que leurs champions étaient trop lents en condition réelle d'utilisation, et ont
perdu du temps sur ce point.

\bigbreak

Concernant la défense, beaucoup ont opté pour valoriser des aliens avec un
environnement proche fermé (des murs ou agents adjacents) qui facilitait la
défense, voire la rendait infaillible. De plus, dans le cas d'aliens très
importants, une stratégie de coopération entre les manchots permettait une
approche défensive solide, en se plaçant sur l'axe d'attaque d'un agent ennemi
pour empêcher qu'il glisse et pousse l'agent qui capture. Il fallait cependant
faire attention à bien choisir à quel moment cela était réellement nécessaire,
et gérer les cas où les attaques ennemies provenaient des deux axes. Enfin,
plusieurs ont remarqué un comportement intéressant à mettre en place lorsque des
agents n'avaient à priori rien à faire durant le tour actuel : les diriger vers
d'autres agents alliés, dans l'idée de les protéger plus tard. Cette stratégie
s'est révelée être une bonne initiative et offrait un support défensif à moyen
terme.

\bigbreak

Pour l'attaque, la plupart des candidats se basaient principalement sur des
heuristiques pour décider des actions à réaliser. La capture étant le but
principal, cette dernière était souvent déterminée en fonction de nombreux
paramètres comme la distance, le score de capture, la présence ou non d'agent
ennemi, ou encore l'environnement. L'action pousser était aussi une composante
majeure de l'attaque, et plusieurs candidats ont tenté de nombreuses méthodes 
afin de déterminer la direction optimale sur laquelle pousser un agent ennemi.
Une stratégie assez pertinente lorsqu'un agent pousse un ennemi était de se
décaler d'une case dans un axe orthogonal à la glissade de l'ennemi, pour éviter
qu'au prochain tour, il glisse à nouveau et pousse l'agent.

\newpage

On notera que plus on monte dans le classement, plus les codes des champions
sont génériques, et ont une couche d'abstraction importante. L'IA vainqueure de
cette édition s'est notamment bien différenciée des autres sur ce point précis.
De plus, nous avons pu observer chez la tête du classement l'implémentation de
stratégies de coopérations entre les agents, qui étaient clairement un avantage
par rapport aux autres champions qui traitaient chaque agent de manière
totalement individuelle et séquentielle.

\bigbreak

Nous espérons que le sujet ainsi que la finale en général vous auront plu. Un
grand bravo à tous les finalistes, et à l'année prochaine !

\end{document}
